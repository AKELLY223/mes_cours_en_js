\documentclass{beamer}
\usepackage{minted}
\usepackage{xcolor}
\usetheme{Madrid}
\setbeamertemplate{navigation symbols}{}% Remove navigation symbols


% Configuration pour minted
\definecolor{bgcolor}{rgb}{0.95, 0.95, 0.92}
\setminted{
    bgcolor=bgcolor,
    frame=lines,
    framesep=2mm,
    linenos=true,
    fontsize=\footnotesize
}

\title{Introduction à JavaScript}
\author{\textbf{KELLY}  Abdoulaye}
\institute{\textit{Mali\_Code}}
\titlegraphic{\includegraphics[scale=0.03]{logo}}

\date{mlcode223@gmail.com}

\begin{document}

\frame{\titlepage}

% Slide 1 : Présentation du langage JavaScript
\begin{frame}{Qu'est-ce que JavaScript ?}
\begin{itemize}
    \item JavaScript est un langage de programmation orienté web.
    \item Il permet d'ajouter de l'interactivité aux pages web.
    \item Il s'exécute côté client dans les navigateurs web.
    \item Compatible avec HTML et CSS pour le développement web.
\end{itemize}
\end{frame}

\section{Installation des outils}
\begin{frame}{Installation des outils}
\begin{itemize}
    \item Navigateur web (Chrome, Firefox).
    \item Éditeur de code (VS Code, Sublime Text).
    \item Console développeur (F12 dans le navigateur).
\end{itemize}
\end{frame}

\section{Syntaxe de base}
\begin{frame}{Syntaxe de base}
\begin{itemize}
    \item Variables : \mintinline{javascript}{let}, \mintinline{javascript}{const}, \mintinline{javascript}{var}.
    \item Types de données : nombres, chaînes, booléens, etc.
    \item Opérateurs : arithmétiques, de comparaison, logiques.
\end{itemize}
\end{frame}

\begin{frame}[fragile]{Exemple de code}
\begin{minted}{javascript}
// Déclaration de variables
let message = "Bonjour, JavaScript!";
const PI = 3.14;

// Affichage dans la console
console.log(message);

// Opérations de base
let a = 5;
let b = 10;
let somme = a + b;
console.log("La somme est : " + somme);
\end{minted}
\end{frame}



\section{Structures de contrôle}
\begin{frame}{Structures de contrôle}
\begin{itemize}
    \item Conditions : \mintinline{javascript}{if}, \mintinline{javascript}{else}, \mintinline{javascript}{switch}.
    \item Boucles : \mintinline{javascript}{for}, \mintinline{javascript}{while}.
\end{itemize}
\end{frame}

\begin{frame}[fragile]{Exemple de conditions}
\begin{minted}{javascript}
let age = 18;

if (age >= 18) {
    console.log("Vous êtes majeur.");
} else {
    console.log("Vous êtes mineur.");
}
\end{minted}
\end{frame}

\begin{frame}[fragile]{Exemple de boucles}
\begin{minted}{javascript}
// Boucle for
for (let i = 0; i < 5; i++) {
    console.log("Itération : " + i);
}

// Boucle while
let j = 0;
while (j < 5) {
    console.log("Itération : " + j);
    j++;
}
\end{minted}
\end{frame}

\section{Résumé de l'Introduction à JavaScript}
\begin{frame}{Résumé de l'Introduction à JavaScript}
\begin{itemize}
    \item Introduction à JavaScript et son écosystème.
    \item Installation des outils nécessaires.
    \item Syntaxe de base : variables, types de données, opérateurs.
    \item Structures de contrôle : conditions et boucles.
\end{itemize}
\end{frame}

\begin{frame}{Questions ?}
\begin{center}
\Large Des questions sur le contenu de l'Introduction à JavaScript ?
\end{center}
\end{frame}

\end{document}
