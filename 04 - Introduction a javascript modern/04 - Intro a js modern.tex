\documentclass{beamer}
\usepackage{minted}
\usepackage{xcolor}
\usetheme{Madrid}
\setbeamertemplate{navigation symbols}{}



% Configuration pour minted
\definecolor{bgcolor}{rgb}{0.95, 0.95, 0.92}
\setminted{
    bgcolor=bgcolor,
    frame=lines,
    framesep=2mm,
    linenos=true,
    fontsize=\footnotesize
}
\title{Introduction à JavaScript Modernes}
\author{\textbf{KELLY}  Abdoulaye}
\institute{\textit{Mali\_Code}}
\date{mlcode223@gmail.com}

\begin{document}

\begin{frame}
    \titlepage
    \end{frame}
    
    \begin{frame}{Plan de l'Introduction à ES6+}
    \tableofcontents
    \end{frame}
    
    \section{Introduction à ES6+}
    \begin{frame}{Introduction à ES6+}
    \begin{itemize}
        \item ES6 (ECMAScript 2015) a introduit de nouvelles fonctionnalités pour JavaScript.
        \item Fonctionnalités clés :
            \begin{itemize}
                \item Fonctions fléchées.
                \item Template literals.
                \item Déstructuration.
                \item Classes.
            \end{itemize}
    \end{itemize}
    \end{frame}
    
    \begin{frame}[fragile]{Fonctions fléchées}
    \begin{minted}{javascript}
    // Fonction traditionnelle
    function addition(a, b) {
        return a + b;
    }
    
    // Fonction fléchée
    const addition = (a, b) => a + b;
    
    console.log(addition(5, 3)); // 8
    \end{minted}
    \end{frame}
    
    \begin{frame}[fragile]{Template literals}
    \begin{minted}{javascript}
    const nom = "Alice";
    const age = 25;
    
    // Utilisation de template literals
    const message = `Bonjour, je m'appelle ${nom} et j'ai ${age} ans.`;
    console.log(message); // "Bonjour, je m'appelle Alice et j'ai 25 ans."
    \end{minted}
    \end{frame}
    
    \begin{frame}[fragile]{Déstructuration}
    \begin{minted}{javascript}
    const personne = { nom: "Alice", age: 25 };
    
    // Déstructuration d'un objet
    const { nom, age } = personne;
    console.log(nom); // "Alice"
    console.log(age); // 25
    
    // Déstructuration d'un tableau
    const nombres = [1, 2, 3];
    const [premier, deuxieme] = nombres;
    console.log(premier); // 1
    \end{minted}
    \end{frame}
    
    \section{Promesses et gestion des erreurs}
    \begin{frame}{Promesses en JavaScript}
    \begin{itemize}
        \item Une promesse représente une valeur qui peut être disponible maintenant, plus tard ou jamais.
        \item États d'une promesse : `pending`, `fulfilled`, `rejected`.
        \item Méthodes : \mintinline{javascript}{then}, \mintinline{javascript}{catch}, \mintinline{javascript}{finally}.
    \end{itemize}
    \end{frame}
    
    \begin{frame}[fragile]{Exemple de promesse}
    \begin{minted}{javascript}
    const maPromesse = new Promise((resolve, reject) => {
        setTimeout(() => {
            const succes = true;
            if (succes) {
                resolve("Opération réussie !");
            } else {
                reject("Erreur !");
            }
        }, 2000);
    });
    
    maPromesse
        .then(resultat => console.log(resultat)) // "Opération réussie !"
        .catch(erreur => console.error(erreur)); // "Erreur !"
    \end{minted}
    \end{frame}
    
    \section{Async/Await}
    \begin{frame}{Async/Await}
    \begin{itemize}
        \item \mintinline{javascript}{async} et \mintinline{javascript}{await} simplifient l'utilisation des promesses.
        \item \mintinline{javascript}{async} déclare une fonction asynchrone.
        \item \mintinline{javascript}{await} attend la résolution d'une promesse.
    \end{itemize}
    \end{frame}
    
    \begin{frame}[fragile]{Exemple de async/await}
    \begin{minted}{javascript}
    async function fetchData() {
        try {
            const reponse = await fetch("https://api.example.com/data");
            const data = await reponse.json();
            console.log(data);
        } catch (erreur) {
            console.error("Erreur :", erreur);
        }
    }
    
    fetchData();
    \end{minted}
    \end{frame}
    
    \section{Requêtes HTTP avec Fetch}
    \begin{frame}{Requêtes HTTP avec Fetch}
    \begin{itemize}
        \item L'API `fetch` permet de faire des requêtes HTTP.
        \item Syntaxe simple pour récupérer des données depuis une API.
        \item Retourne une promesse.
    \end{itemize}
    \end{frame}
    
    \begin{frame}[fragile]{Exemple de fetch}
    \begin{minted}{javascript}
    fetch("https://api.example.com/data")
        .then(reponse => reponse.json())
        .then(data => console.log(data))
        .catch(erreur => console.error("Erreur :", erreur));
    \end{minted}
    \end{frame}
    
    \section{Résumé du Jour 4}
    \begin{frame}{Résumé du Jour 4}
    \begin{itemize}
        \item Fonctionnalités ES6+ : fonctions fléchées, template literals, déstructuration.
        \item Promesses : gestion des opérations asynchrones.
        \item Async/Await : simplification des promesses.
        \item Requêtes HTTP avec `fetch`.
    \end{itemize}
    \end{frame}
    
    \begin{frame}{Questions ?}
    \begin{center}
    \Large Des questions sur le contenu de l'in ?
    \end{center}
    \end{frame}
    \end{document}