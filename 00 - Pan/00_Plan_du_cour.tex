\documentclass[12pt]{article}

\usepackage[a4paper, margin=1in]{geometry} % Configuration de la page
\usepackage{fancyhdr} % Pour gérer l'en-tête
\usepackage[utf8]{inputenc}
\usepackage{enumitem}
\usepackage{graphicx} % Pour les graphiques si besoin
\usepackage{amsmath} 

\pagestyle{fancy} % Style personnalisé pour l'en-tête
\fancyhf{} % Efface les en-têtes et pieds de page

% Personnalisation de l'en-tête
\fancyhead[L]{\textbf{MaliCode}\\ Filière : Dev Web} % Gauche
\fancyhead[R]{\textbf{Classe:  A1}\\ Module: JavaScript} % Droite

% Barre horizontale sous l'en-tête
\renewcommand{\headrulewidth}{0.5pt}
\begin{document}
% Titre centré sous l'en-tête
\vspace*{0.3cm} % Espacement depuis l'en-tête
\begin{center}
    \title{Plan du Cours : JavaScript}
    \date{\today}
\end{center}

\section*{Objectifs du cours}
\begin{itemize}
    \item Comprendre les fondamentaux de JavaScript.
    \item Maîtriser la manipulation du DOM et les événements.
    \item Découvrir les concepts avancés comme les promesses, async/await, et les modules.
    \item Apprendre à interagir avec des APIs externes.
    \item Initiation à des frameworks/libraries populaires (optionnel).
\end{itemize}

\section*{Semaine 1 : Introduction à JavaScript (3 heures)}
\subsection*{Présentation de JavaScript }
\begin{itemize}
    \item Historique et rôle de JavaScript.
    \item Différence entre JavaScript et autres langages.
    \item Environnements d'exécution (navigateur, Node.js).
\end{itemize}

\subsection*{Bases du langage }
\begin{itemize}
    \item Syntaxe de base (variables, commentaires).
    \item Types de données (nombres, chaînes, booléens, objets, tableaux).
    \item Opérateurs (arithmétiques, de comparaison, logiques).
\end{itemize}

\section*{Semaine 2 : Structures de contrôle et fonctions (3 heures)}
\subsection*{Structures de contrôle }
\begin{itemize}
    \item Conditions (if, else, switch).
    \item Boucles (for, while, do...while).
\end{itemize}

\subsection*{Fonctions }
\begin{itemize}
    \item Déclaration et invocation de fonctions.
    \item Paramètres et valeurs de retour.
    \item Portée des variables (let, const, var).
\end{itemize}

\section*{Semaine 3 : Manipulation des tableaux et objets (3 heures)}
\subsection*{Tableaux }
\begin{itemize}
    \item Méthodes courantes (push, pop, map, filter, reduce).
    \item Parcourir un tableau.
\end{itemize}

\subsection*{Objets }
\begin{itemize}
    \item Création et manipulation d'objets.
    \item Accéder aux propriétés.
    \item Méthodes d'objets.
\end{itemize}

\section*{Semaine 4 : Introduction au DOM (3 heures)}
\subsection*{Qu'est-ce que le DOM ? }
\begin{itemize}
    \item Structure du DOM.
    \item Accéder aux éléments (getElementById, querySelector).
\end{itemize}

\subsection*{Manipulation du DOM }
\begin{itemize}
    \item Modifier le contenu et les styles.
    \item Ajouter/supprimer des éléments.
\end{itemize}

\section*{Semaine 5 : Événements et interactions (3 heures)}
\subsection*{Gestion des événements }
\begin{itemize}
    \item Écouteurs d'événements (addEventListener).
    \item Types d'événements (clic, souris, clavier).
\end{itemize}

\subsection*{Formulaires et validation }
\begin{itemize}
    \item Interagir avec les formulaires.
    \item Validation basique.
\end{itemize}

\section*{Semaine 6 : Programmation asynchrone (3 heures)}
\subsection*{Callbacks }
\begin{itemize}
    \item Fonctions de rappel et leur utilisation.
\end{itemize}

\subsection*{Promesses}
\begin{itemize}
    \item Création et gestion des promesses.
    \item Chaînage de promesses.
\end{itemize}

\subsection*{Async/Await }
\begin{itemize}
    \item Syntaxe et utilisation.
\end{itemize}

\section*{Semaine 7 : Manipulation des APIs (3 heures)}
\subsection*{Requêtes HTTP }
\begin{itemize}
    \item Utilisation de fetch pour les requêtes AJAX.
    \item Comprendre les réponses JSON.
\end{itemize}

\subsection*{Utilisation d'une API publique }
\begin{itemize}
    \item Exemple avec une API REST (ex : OpenWeatherMap, GitHub).
\end{itemize}

\section*{Semaine 8 : Modules et outils modernes (3 heures)}
\subsection*{Modules JavaScript }
\begin{itemize}
    \item Import/export de modules.
    \item Utilisation dans un projet.
\end{itemize}

\subsection*{Outils de développement }
\begin{itemize}
    \item Présentation de npm/yarn.
    \item Utilisation de ESLint et Prettier.
\end{itemize}

\section*{Semaine 9 : Introduction aux frameworks (3 heures)}
\subsection*{Présentation des frameworks }
\begin{itemize}
    \item Vue.js, React, ou Angular (au choix).
\end{itemize}

\subsection*{Initiation à un framework }
\begin{itemize}
    \item Installation et premier projet.
    \item Composants et état.
\end{itemize}

\section*{Semaine 10 : Projet pratique (3 heures)}
\subsection*{Conception d'un projet }
\begin{itemize}
    \item Définition des objectifs et planification.
\end{itemize}

\subsection*{Développement guidé }
\begin{itemize}
    \item Implémentation des fonctionnalités.
\end{itemize}

\section*{Semaine 11 : Bonnes pratiques et débogage (3 heures)}
\subsection*{Bonnes pratiques de codage }
\begin{itemize}
    \item Code propre, commentaires, conventions.
\end{itemize}

\subsection*{Débogage }
\begin{itemize}
    \item Utilisation des outils de développement du navigateur.
\end{itemize}

\section*{Semaine 12 : Révision et examen (3 heures)}
\subsection*{Révision des concepts clés .}
\subsection*{Examen pratique ou présentation de projet .}


\fancyfoot[L]{Mr Kelly} % Texte à gauche
\fancyfoot[R]{+223 91 75 54 37} % Texte à droite
\renewcommand{\footrulewidth}{0.4pt} % Ligne fine au-dessus du pied de page

\end{document}