\documentclass{beamer}
\usepackage{minted}
\usepackage{xcolor}
\usetheme{Madrid}
\setbeamertemplate{navigation symbols}{}% Remove navigation symbols


% Configuration pour minted
\definecolor{bgcolor}{rgb}{0.95, 0.95, 0.92}
\setminted{
    bgcolor=bgcolor,
    frame=lines,
    framesep=2mm,
    linenos=true,
    fontsize=\footnotesize
}
\title{Fonctions et objets en JavaScript}
\author{\textbf{KELLY}  Abdoulaye}
\institute{\textit{Mali\_Code}}
\titlegraphic{\includegraphics[scale=0.03]{logo}}

\date{mlcode223@gmail.com}
\begin{document}

\begin{frame}
    \titlepage
\end{frame}
    
\begin{frame}{Plan du Jour 2}
    \tableofcontents
\end{frame}



\section{Fonctions en JavaScript}
\begin{frame}{Fonctions en JavaScript}
\begin{itemize}
    \item Une fonction est un bloc de code réutilisable.
    \item Déclaration avec \mintinline{javascript}{function} ou \mintinline{javascript}{const}.
    \item Passage de paramètres et retour de valeurs.
\end{itemize}
\end{frame}

\begin{frame}[fragile]{Exemple de fonction}
\begin{minted}{javascript}
// Déclaration d'une fonction
function direBonjour(nom) {
    return "Bonjour, " + nom + "!";
}

// Appel de la fonction
let message = direBonjour("Alice");
console.log(message); // "Bonjour, Alice!"
\end{minted}
\end{frame}

\begin{frame}[fragile]{Fonctions fléchées (ES6+)}
\begin{minted}{javascript}
// Fonction fléchée
const addition = (a, b) => {
    return a + b;
};

// Version raccourcie
const multiplication = (a, b) => a * b;

console.log(addition(5, 3)); // 8
console.log(multiplication(4, 2)); // 16
\end{minted}
\end{frame}

\section{Portée des variables}
\begin{frame}{Portée des variables}
\begin{itemize}
    \item \mintinline{javascript}{let} : portée de bloc.
    \item \mintinline{javascript}{const} : portée de bloc, valeur constante.
    \item \mintinline{javascript}{var} : portée de fonction (obsolète).
\end{itemize}
\end{frame}

\begin{frame}[fragile]{Exemple de portée}
\begin{minted}{javascript}
if (true) {
    let x = 10; // Portée de bloc
    const y = 20; // Portée de bloc
    var z = 30; // Portée de fonction
}

console.log(z); // 30 (var est accessible)
console.log(x); // Erreur : x n'est pas défini
console.log(y); // Erreur : y n'est pas défini
\end{minted}
\end{frame}

\section{Objets en JavaScript}
\begin{frame}{Objets en JavaScript}
\begin{itemize}
    \item Un objet est une collection de propriétés (clé-valeur).
    \item Les propriétés peuvent être des fonctions (méthodes).
    \item Syntaxe : \mintinline{javascript}{{ clé: valeur }}.
\end{itemize}
\end{frame}

\begin{frame}[fragile]{Exemple d'objet}
\begin{minted}{javascript}
// Création d'un objet
const personne = {
    nom: "Alice",
    age: 25,
    saluer: function() {
        console.log("Bonjour, je m'appelle " + this.nom);
    }
};

// Accès aux propriétés
console.log(personne.nom); // "Alice"
personne.saluer(); // "Bonjour, je m'appelle Alice"
\end{minted}
\end{frame}

\section{Tableaux et méthodes courantes}
\begin{frame}{Tableaux en JavaScript}
\begin{itemize}
    \item Un tableau est une liste ordonnée d'éléments.
    \item Méthodes courantes : \mintinline{javascript}{push}, \mintinline{javascript}{pop}, \mintinline{javascript}{map}, \mintinline{javascript}{filter}.
\end{itemize}
\end{frame}

\begin{frame}[fragile]{Exemple de tableau}
\begin{minted}{javascript}
// Création d'un tableau
let fruits = ["pomme", "banane", "orange"];

// Ajout d'un élément
fruits.push("kiwi");

// Suppression du dernier élément
fruits.pop();

// Transformation avec map
let longueurs = fruits.map(fruit => fruit.length);
console.log(longueurs); // [5, 6, 6]
\end{minted}
\end{frame}

\section{Résumé des Fonctions et objets en JavaScript}
\begin{frame}{ Résumé des Fonctions et objets en JavaScript}
\begin{itemize}
    \item Fonctions : déclaration, paramètres, retour.
    \item Portée des variables : \mintinline{javascript}{let}, \mintinline{javascript}{const}, \mintinline{javascript}{var}.
    \item Objets : propriétés et méthodes.
    \item Tableaux : manipulation et méthodes courantes.
\end{itemize}
\end{frame}

\begin{frame}{Questions ?}
\begin{center}
\Large Des questions sur le contenu des Fonctions et objets en JavaScript ?
\end{center}
\end{frame}

\end{document}
