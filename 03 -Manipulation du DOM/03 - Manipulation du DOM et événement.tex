\documentclass{beamer}
\usepackage{minted}
\usepackage{xcolor}
\usetheme{Madrid}
\setbeamertemplate{navigation symbols}{}


% Configuration pour minted
\definecolor{bgcolor}{rgb}{0.95, 0.95, 0.92}
\setminted{
    bgcolor=bgcolor,
    frame=lines,
    framesep=2mm,
    linenos=true,
    fontsize=\footnotesize
}
\title{Manipulation du DOM et événements}
\author{\textbf{KELLY}  Abdoulaye}
\institute{\textit{Mali\_Code}}
\date{mlcode223@gmail.com}
\begin{document}

\begin{frame}
    \titlepage
    \end{frame}
    
    \begin{frame}{Manipulation du DOM et événements}
    \tableofcontents
    \end{frame}
    
    \section{Introduction au DOM}
    \begin{frame}{Introduction au DOM}
    \begin{itemize}
        \item Le DOM (Document Object Model) représente la structure d'une page HTML.
        \item JavaScript permet de manipuler le DOM pour modifier le contenu, la structure et le style d'une page web.
        \item Accéder aux éléments avec des méthodes comme \mintinline{javascript}{getElementById}, \mintinline{javascript}{querySelector}.
    \end{itemize}
    \end{frame}
    
    \begin{frame}[fragile]{Exemple d'accès au DOM}
    \begin{minted}{html}
    <!-- Exemple de structure HTML -->
    <div id="contenu">
        <p id="message">Bonjour, monde !</p>
        <button id="bouton">Cliquez-moi</button>
    </div>
    \end{minted}
    
    \begin{minted}{javascript}
    // Accéder à un élément par son ID
    let message = document.getElementById("message");
    console.log(message.textContent); // "Bonjour, monde !"
    \end{minted}
    \end{frame}
    
    \section{Manipulation du DOM}
    \begin{frame}{Manipulation du DOM}
    \begin{itemize}
        \item Modifier le contenu : \mintinline{javascript}{textContent}, \mintinline{javascript}{innerHTML}.
        \item Modifier les styles : \mintinline{javascript}{style}.
        \item Ajouter ou supprimer des éléments : \mintinline{javascript}{appendChild}, \mintinline{javascript}{removeChild}.
    \end{itemize}
    \end{frame}
    
    \begin{frame}[fragile]{Exemple de manipulation du DOM}
    \begin{minted}{javascript}
    // Modifier le contenu d'un élément
    let message = document.getElementById("message");
    message.textContent = "Nouveau message !";
    
    // Modifier le style
    message.style.color = "red";
    message.style.fontSize = "20px";
    
    // Ajouter un nouvel élément
    let nouveauParagraphe = document.createElement("p");
    nouveauParagraphe.textContent = "Ceci est un nouveau paragraphe.";
    document.getElementById("contenu").appendChild(nouveauParagraphe);
    \end{minted}
    \end{frame}
    
    \section{Gestion des événements}
    \begin{frame}{Gestion des événements}
    \begin{itemize}
        \item Les événements permettent de réagir aux actions de l'utilisateur (clics, saisie, etc.).
        \item Méthodes : \mintinline{javascript}{addEventListener}.
        \item Types d'événements : \mintinline{javascript}{click}, \mintinline{javascript}{mouseover}, \mintinline{javascript}{keydown}, etc.
    \end{itemize}
    \end{frame}
    
    \begin{frame}[fragile]{Exemple de gestion d'événement}
    \begin{minted}{javascript}
    // Ajouter un écouteur d'événement
    let bouton = document.getElementById("bouton");
    bouton.addEventListener("click", function() {
        alert("Vous avez cliqué sur le bouton !");
    });
    \end{minted}
    \end{frame}
    
    \section{Validation de formulaire}
    \begin{frame}{Validation de formulaire}
    \begin{itemize}
        \item JavaScript permet de valider les données saisies dans un formulaire avant soumission.
        \item Exemple : vérifier si un champ est vide ou si un email est valide.
    \end{itemize}
    \end{frame}
    
    \begin{frame}[fragile]{Exemple de validation de formulaire}
    \begin{minted}{html}
    <form id="monFormulaire">
        <input type="text" id="nom" placeholder="Votre nom" required>
        <input type="email" id="email" placeholder="Votre email" required>
        <button type="submit">Soumettre</button>
    </form>
    <p id="erreur" style="color: red;"></p>
    \end{minted}
    
    \begin{minted}{javascript}
    document.getElementById("monFormulaire").addEventListener("submit", function(event) {
        let nom = document.getElementById("nom").value;
        let email = document.getElementById("email").value;
    
        if (nom === "" || email === "") {
            document.getElementById("erreur").textContent = "Tous les champs sont obligatoires !";
            event.preventDefault(); // Empêche la soumission du formulaire
        }
    });
    \end{minted}
    \end{frame}
    
    \section{Résumé sur Manipulation du DOM et événements}
    \begin{frame}{Résumé sur Manipulation du DOM et événements}
    \begin{itemize}
        \item Introduction au DOM et accès aux éléments.
        \item Manipulation du DOM : contenu, styles, ajout/suppression d'éléments.
        \item Gestion des événements : clics, saisie, etc.
        \item Validation de formulaire avec JavaScript.
    \end{itemize}
    \end{frame}
    
    \begin{frame}{Questions ?}
    \begin{center}
    \Large Des questions sur le contenu Manipulation du DOM et événements ?
    \end{center}
    \end{frame}

\end{document}
