\documentclass{beamer}
\usepackage{minted}
\usepackage{xcolor}
\usetheme{Madrid}
\setbeamertemplate{navigation symbols}{}



% Configuration pour minted
\definecolor{bgcolor}{rgb}{0.95, 0.95, 0.92}
\setminted{
    bgcolor=bgcolor,
    frame=lines,
    framesep=2mm,
    linenos=true,
    fontsize=\footnotesize
}
\title{Projet pratique et révision}
\institute{Mali\_Code}
\date{12 février 2025}
\author{ \textbf{KELLY} Abdoulaye}

\begin{document}

\begin{frame}
    \titlepage
    \end{frame}
    
    \begin{frame}{Plan Projet pratique et révision}
    \tableofcontents
    \end{frame}
    
    \section{Projet pratique}
    \begin{frame}{Projet pratique}
    \begin{itemize}
        \item Création d'une application web simple.
        \item Intégration des concepts appris pendant le cours :
            \begin{itemize}
                \item Manipulation du DOM.
                \item Gestion des événements.
                \item Requêtes HTTP avec `fetch`.
                \item Utilisation de `async/await`.
            \end{itemize}
        \item Travail en petits groupes ou individuellement.
    \end{itemize}
    \end{frame}
    
    \begin{frame}[fragile]{Exemple de projet}
    \begin{minted}{html}
    <!-- Structure HTML -->
    <div id="app">
        <h1>Générateur de citations</h1>
        <button id="bouton">Générer une citation</button>
        <p id="citation"></p>
    </div>
    \end{minted}
    
    \begin{minted}{javascript}
    // JavaScript
    const bouton = document.getElementById("bouton");
    const citation = document.getElementById("citation");
    bouton.addEventListener("click", async () => {
    try {
        const reponse = await fetch("https://api.quotable.io/random");
         const data = await reponse.json();
        citation.textContent = data.content;
    } catch (erreur) {
    citation.textContent = "Erreur lors de la récupération de la citation.";
        }
    });
    \end{minted}
    \end{frame}
    
    \section{Révision des concepts clés}
    \begin{frame}{Révision des concepts clés}
    \begin{itemize}
        \item \textbf{JavaScript de base} : variables, fonctions, structures de contrôle.
        \item \textbf{Manipulation du DOM} : sélection, modification, gestion des événements.
        \item \textbf{JavaScript moderne} : ES6+, promesses, `async/await`.
        \item \textbf{Requêtes HTTP} : utilisation de `fetch`.
    \end{itemize}
    \end{frame}
    
    \begin{frame}[fragile]{Rappel sur les promesses}
    \begin{minted}{javascript}
    const maPromesse = new Promise((resolve, reject) => {
        setTimeout(() => {
            const succes = true;
            if (succes) {
                resolve("Succès !");
            } else {
                reject("Échec !");
            }
        }, 1000);
    });
    
    maPromesse
        .then(resultat => console.log(resultat))
        .catch(erreur => console.error(erreur));
    \end{minted}
    \end{frame}
    
    \begin{frame}[fragile]{Rappel sur async/await}
    \begin{minted}{javascript}
    async function fetchData() {
        try {
            const reponse = await fetch("https://api.example.com/data");
            const data = await reponse.json();
            console.log(data);
        } catch (erreur) {
            console.error("Erreur :", erreur);
        }
    }
    
    fetchData();
    \end{minted}
    \end{frame}
    
    \section{Conseils pour approfondir}
    \begin{frame}{Conseils pour approfondir}
    \begin{itemize}
        \item Pratiquer régulièrement sur des plateformes comme :
            \begin{itemize}
                \item \textbf{freeCodeCamp} : https://www.freecodecamp.org/
                \item \textbf{LeetCode} : https://leetcode.com/
                \item \textbf{Codewars} : https://www.codewars.com/
            \end{itemize}
        \item Explorer des frameworks JavaScript comme React, Vue.js ou Angular.
        \item Contribuer à des projets open source sur GitHub.
    \end{itemize}
    \end{frame}
    
    \section{Questions et réponses}
    \begin{frame}{Questions et réponses}
    \begin{center}
    \Large Une séance de questions-réponses pour clarifier les derniers points.
    \end{center}
    \end{frame}
    
    \section{Résumé du cours}
    \begin{frame}{Résumé du cours}
    \begin{itemize}
        \item Introduction à JavaScript et à son écosystème.
        \item Manipulation du DOM et gestion des événements.
        \item Concepts modernes : ES6+, promesses, `async/await`.
        \item Requêtes HTTP avec `fetch`.
        \item Réalisation d'un projet pratique.
    \end{itemize}
    \end{frame}
    
    \begin{frame}{Merci !}
    \begin{center}
    \Large Merci pour votre participation et bon courage pour la suite !
    \end{center}
    \end{frame}
    \end{document}